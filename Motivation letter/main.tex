%% start of file `template.tex'.
%% Copyright 2006-2013 Xavier Danaux (xdanaux@gmail.com).
%
% This work may be distributed and/or modified under the
% conditions of the LaTeX Project Public License version 1.3c,
% available at http://www.latex-project.org/lppl/.


\documentclass[11pt,a4paper,sans]{moderncv}        % possible options include font size ('10pt', '11pt' and '12pt'), paper size ('a4paper', 'letterpaper', 'a5paper', 'legalpaper', 'executivepaper' and 'landscape') and font family ('sans' and 'roman')

% moderncv themes
\moderncvstyle{banking}                            % style options are 'casual' (default), 'classic', 'oldstyle' and 'banking'
\moderncvcolor{red}                                % color options 'blue' (default), 'orange', 'green', 'red', 'purple', 'grey' and 'black'
%\renewcommand{\familydefault}{\sfdefault}         % to set the default font; use '\sfdefault' for the default sans serif font, '\rmdefault' for the default roman one, or any tex font name
%\nopagenumbers{}                                  % uncomment to suppress automatic page numbering for CVs longer than one page

% character encoding
\usepackage[utf8]{inputenc}                       % if you are not using xelatex ou lualatex, replace by the encoding you are using
%\usepackage{CJKutf8}                              % if you need to use CJK to typeset your resume in Chinese, Japanese or Korean

% adjust the page margins
\usepackage[scale=0.80]{geometry}
%\setlength{\hintscolumnwidth}{3cm}                % if you want to change the width of the column with the dates
%\setlength{\makecvtitlenamewidth}{10cm}           % for the 'classic' style, if you want to force the width allocated to your name and avoid line breaks. be careful though, the length is normally calculated to avoid any overlap with your personal info; use this at your own typographical risks...

% personal data
\name{ShiHong}{Lee}
\title{Institute for Marine and Antarctic Studies, UTas}                               % optional, remove / comment the line if not wanted
\address{20 Castray Esplanade}{Battery Point TAS 7004}{Australia}% optional, remove / comment the line if not wanted; the "postcode city" and and "country" arguments can be omitted or provided empty
\phone[mobile]{+61 3 6226 6379}                   % optional, remove / comment the line if not wanted
\email{shihongl@utas.edu.au}                               % optional, remove / comment the line if not wanted
%\homepage{www.johndoe.com}                         % optional, remove / comment the line if not wanted
%\extrainfo{additional information}                 % optional, remove / comment the line if not wanted
%\photo[64pt][0.4pt]{picture}                       % optional, remove / comment the line if not wanted; '64pt' is the height the picture must be resized to, 0.4pt is the thickness of the frame around it (put it to 0pt for no frame) and 'picture' is the name of the picture file
%\quote{Some quote}                                 % optional, remove / comment the line if not wanted

% to show numerical labels in the bibliography (default is to show no labels); only useful if you make citations in your resume
%\makeatletter
%\renewcommand*{\bibliographyitemlabel}{\@biblabel{\arabic{enumiv}}}
%\makeatother
%\renewcommand*{\bibliographyitemlabel}{[\arabic{enumiv}]}% CONSIDER REPLACING THE ABOVE BY THIS

% bibliography with mutiple entries
%\usepackage{multibib}
%\newcites{book,misc}{{Books},{Others}}
%----------------------------------------------------------------------------------
%            content
%----------------------------------------------------------------------------------
\begin{document}
%-----       letter       ---------------------------------------------------------
% recipient data
\recipient{Dr. Alexandra Kraberg}{The Alfred Wegener Institute, Helmholtz Centre for Polar and Marine Research}
\date{\today}
\opening{Dear Dr. Kraberg,}
\closing{Yours sincerely,}
%\enclosure[Attached]{curriculum vit\ae{}}          % use an optional argument to use a string other than "Enclosure", or redefine \enclname
\makelettertitle
I am writing to express my motivation to participate in and to describe my suitability for the Joint AWI-SAHFOS summer school - ``The importance of time series observations for the assessment of the biological and societal impacts of climate change". \linebreak\newline
I commenced a PhD at the University of Tasmania in January of 2013, with Prof. Andrew McMinn as my major advisor. The core focus of my PhD is on the detection and quantification of climate change impacts on micro-phytobenthic communities in temperate and tropical climes. My primary interests are in biological time series analysis and the investigation of the impacts of increased climate change in remote and sensitive estuarine regions.  \linebreak\newline
My recent publication described the behavioural and physiological performance of micro-phytobenthos in response to extreme temperatures - see my attached CV for details. My current PhD publication describes the annual cycles of primary production and photosynthesis of micro-phytobenthos in the southern temperate regions. I will present this paper and use this dataset in the data surgeries and student projects at the Joint AWI-SAHFOS summer school.  \linebreak\newline
I have strong skills in data collection, coastal oceanography, phycology, and biochemistry and I believe that by attending this summer school I will strengthen my research in the areas of time-series analysis and statistics along with developing essential skills in data management and statistics. \linebreak\newline
From the lecture series I want to complement my existing knowledge with skills and techniques for taking my long-term observational data and using it to assess the biological and societal impact of climate change in coastal marine systems. I also wish to take the opportunity to initiate collaborations and dialog with the experts from AWI and SAHFOS teaching the course. \linebreak\newline
Coming mid-way through my PhD candidature, the lecture series will have maximum benefit for my doctoral studies allowing me to integrate the knowledge and techniques presented immediately into my research. \linebreak\newline
\makeletterclosing
\end{document}
%% end of file `template.tex'.
